\documentclass[main]{subfiles}
\begin{document}

%@@@@@@@@@@@@@@@@@@@@@@@@@@@@@@
% Main Topics: linear programming duality, complementary slackness 
% Duality theorem for linear programing - 09.10.2017
% author: Vanessa Leite

\section{Linear Programming Duality}

\textbf{Theorem (Neuman '47): Let $P = \{x \in \mathbb{R}^n \mid Ax \leq b \}
\neq \emptyset$. Assume there exists a point in the polyhedron. $D = \{ y \in
\mathbb{R}^m \mid y \geq 0$, $y^TA = c^T\} \neq \emptyset$.
\emph{$\displaystyle \max_{x \in \mathcal{P}} c^{T}x = \min_{y \in D} y^T b$ }}

\subparagraph{Proof:}
\begin{itemize}
\item $D$ is a polyhedron in standard form, $D \neq \emptyset$, so, there is no
line, then, $D$ has an extreme point $x^* \in \mathcal{P}$, because
$\mathcal{P} \neq \emptyset$. 
$c^T x \underbrace{=}_{\forall y \in \mathcal{D}} \underbrace{y^T}_{\geq 0}
\underbrace{Ax}_{\leq b} \leq y^T b$, therefore, the minimum is not infinity.
$\rightarrow \exists y^* \in \mathcal{D}$, extreme point such that
$\underbrace{\delta^*}_{\text{optimal value}} = y^{*^T} b = min\{y^T b \mid y
\in \mathcal{D}\}$
\item \textbf{$y^*$ exists.}
$\forall x \in \mathcal{P}$, then $c^T x = y^{*^T}Ax \leq \delta^* \rightarrow
\displaystyle \max_{x \in \mathcal{P}} c^{T}x$ is bounded (can not go to
$+\infty$).
\end{itemize}

Show $\{x \in\mathcal{P} \mid c^T x \geq \delta^* \} \neq \emptyset$. Suppose
it is empty, then:\\
$\rightarrow Ax \leq b$ has no solution \\
$-c^T x \leq -\delta^*$

By Farka's Lemma, there exists multipliers $z \in \mathbb{R}_{+}^{m}$ and
$\lambda \in \mathbb{R}_{+}$, such that $z^T A - \lambda c^T = 0$, $z^T b -
\lambda \delta^* < 0$.\\
Assume $\lambda = 0$: $\exists$ solution $z \geq 0$ such that $z^T A = 0$,
$z^T b < 0 \iff \mathcal{P} = \emptyset$.
Assume $\lambda > 0$: define $y = \frac{z}{\lambda}$. Then, $z^T A = \lambda 
c^T \iff y^T A = c^T$, $z^T b < \lambda \delta^* \iff y^T b < \delta^*$.

$\delta^*$ is $\displaystyle \min_{y \in \mathcal{D}} y^{T}b$, so it is a
contradiction $y^T b < \delta^* \rightarrow \{ x \in \mathcal{P} \mid c^T x 
\geq \delta^*\} \neq \emptyset$.

\subsection{Weak complementary slackness}
\textbf{Theorem: $P = \{x \in \mathbb{R}^n \mid Ax \leq b \} \neq \emptyset$
and $D = \{ y \geq 0 \mid y^T A = c^T \} \neq \emptyset$. We say $x \in
\mathcal{P}$ and $y \in \mathcal{D}$ are simultaneously optimal solutions iff
$y_i (A_{i\cdot}x - b_i) = 0$ $\forall i = 1, \dots, m$. }

\subparagraph{Proof:}
\begin{equation} \label{eq:proof-weak-slackness}
c^T x = y^T Ax = \sum_{i=1}^{m} y_i A_{i\cdot} x = \sum_{i: y_i > 0} y_i A_{i
\cdot} x \leq \sum_{i: y_i > 0} y_i b_i = y^T b
\end{equation}

$x \in P, y \in D$ are simultaneously optimal $\underbrace{\text{iff}}_{duality} c^T x = y^T b \iff \forall i: y_i > 0$, then $A_i x = b_i$.


\end{document}