\documentclass[main]{subfiles}
\begin{document}

%@@@@@@@@@@@@@@@@@@@@@@@@@@@@@@
% Main Topics: 
% Algorithms for hard problems: knapsack problem - 20.11.2017
% author: Vanessa Leite

\section{Algorithms for hard problems}

\subsection{knapsack problem}
\paragraph{Definition - knapsack problem}

Given a set $C = \{ c_{1}, \dots, c_{n} \} \subseteq \mathbb{Z}_{+} \setminus \{ 0\}$ (values) and $A = \{a_{1}, \dots, a_{n} \subseteq \mathbb{Z}_{+} \setminus \{ 0\} \}$ (weights), $a_{0} \in \mathbb{Z}_{+} \setminus \{ 0\}, a_{0} \geq a_{i} \forall i$.

$\gamma(k, a_{0}) = \{ \max \sum_{i =1}^{k} c_{i} x_{i} \mid \sum_{i=1}^{k} a_{i}x_{i} \leq a_{0}, x \in \{0,1\}^{k} \}$ is a (binary) knapsack problem. (If $x \in \mathbb{Z}_{+}^{k}$, then it is a integer knapsack problem).

Dual:
$\alpha(k, c_{0}) = \{ \min \sum_{i =1}^{k} a_{i} x_{i} \mid \sum_{i=1}^{k} c_{i}x_{i} \geq c_{0}, x \in \{0,1\}^{k} \}$

\paragraph{A k-cursive algorithm}

\[
  \gamma(1, a_{0})=\begin{cases}
               0, a_{1} > a_{0}\\
               c_{1}, a_{1} \leq a_{0}\\
            \end{cases}
\]

$\gamma(k, a_{0}) = max\{\gamma(k-1, a_{0}), \gamma(k-1, a_{0} - a_{k}) + c_{k}\}$

The number of iterations for a given $a_{0}$ is $\mathcal{O}(n \times a_{0})$.

For the dual knapsack problem:
\[
  \alpha(1, c_{0})=\begin{cases}
               +\infty, c_{1} < c_{0}\\
               a_{1}, otherwise\\
            \end{cases}
\]

$\alpha(k, c_{0}) = mix\{\alpha(k-1, c_{0}), \alpha(k-1, c_{0} - c_{k}) + a_{k}\}$

\paragraph{Corollary} The knapsack problem can be solved in time $\mathcal{O}(n \times a_{0})$ and the dual version in time $\mathcal{O}(n \times c_{0}) = \mathcal{O}(n^{2} \times max\{c_{i}: i = \{1, \dots, n\}\})$

\paragraph{Notation} For numbers $d_{1}, \dots, d_{k}$, $d_{max} = max\{d_{i}: i =\{1, \dots, k\} \}$

\subsection{Connections primal/dual knapsack}

\paragraph{Lemma} $\alpha(k, \gamma(k, a_{0}) \leq a_{0}$

\subparagraph{Proof:}
Let $x^{*} \in \{0,1\}^{k}$ attain optimal value for $\gamma(k, a_{0})$.
Then:
\begin{itemize}
\item (i) $\sum a_{i}x_{i}^{*} \leq a_{0}$
\item (ii) $\sum c_{i}x_{i}^{*} = \gamma(k, a_{0})$
\end{itemize}

Consider the dual: $\min \{\sum_{i=1}^{k} a_i x_{i} \mid \sum_{i=1}^{k} c_{i}x_{i} \geq \gamma(k, a_{0})\}$, $x^{*}$ is feasible for the dual, $\leq \sum a_{i}x_{i}^{*} \leq a_{0}$.

\paragraph{Lemma} Let $\mathbb{A}$ be an algorithm for computing $\alpha(k, c_{0}) \forall k$ and $c_{0}$. With "polynomial" many calls of $\mathbb{A}$ we can compute $\gamma(k, a_{0})$.

\subparagraph{Proof:}
$\gamma(k, a_{0}) = \max c^{T}x, a^{T}x \leq a_{0}, x \in \{0,1\}^{k}$. Assume $a_{0} \geq a_{max}$:

$\underbrace{1}_{\underline{\mu}} \leq \gamma(k, a_{0}) \leq \underbrace{\sum_{i=1}^{k}c_{i}}_{\bar{\mu}}$.

Consider $\mu = \frac{\bar{\mu} + \underline{\mu}}{2}$. Compute $\alpha(k, \mu)$, then two things ca happen:
\begin{itemize}
\item (i) the result is greather than $a_{0} \rightarrow \bar{\mu} = \mu$
\item (ii) the result is less or equal $a_{0} \rightarrow \underline{\mu} = \mu$

The number of steps that takes until $\underline{\mu} = \bar{\mu}$ is $log(k \times c_{max})$.

\paragraph{Remark}
\end{itemize}

\subsection{Two approximation results}

\end{document}
