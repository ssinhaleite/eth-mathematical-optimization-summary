\documentclass[main]{subfiles}
\begin{document}

%@@@@@@@@@@@@@@@@@@@@@@@@@@@@@@
% Main Topics: basis exchange and pivoting primal simplex algorithm
% The Simplex Algorithm II - 19.10.2017
% author: Vanessa Leite

\section{Simplex Algorithm II}

Consider $\min \{c^T x\mid Ax = b, x\geq 0\}$ as the primal and $\max \{b^T y
\mid A^T y \leq c\}$ as the dual.

What is a basis? A subset of the variables with size $N$.
A subset of the columns with size $N$, they are rows linearly independent, i.e.,
$A \in \mathbb{R}^{m \times n}$ (full row rank), $B$ basis, $B \subseteq \{1,
\dots, m\}$, $|B| = n$.

\paragraph{$A^{-1}_B$ exists.}
Given a basis it can be associated with two points $B \rightarrow
\begin{pmatrix}
x^*\\
y^*
\end{pmatrix}$.
They can be feasible or not. $(x^*_B, x^*_N) = (A^{-1}_B b, 0)$ and
$y^*_B = c^T_B A^{-1}_B$.\\

$B$ is optimal basis if $\bar{b}\geq 0$ and $\bar{c} \geq 0$.\\

Simplex algorithm is jumping between basis, geometrically jumping between
vertices. This is jumping basis because, every time we change the selected
vertex, we change the variables of the basis.

\paragraph{Definition: Given $x^* \in \mathcal{P}$, $d \in \mathbb{R}^n$.
$d$ is a feasible direction if we can move along the vector $d$, i.e, if there
exists $\lambda > 0$ such that $x^* + \lambda d \in \mathcal{P}$.}
$\lambda$ needs to be greather than zero to move far from $x^*$, but this can
be a problem in degeneracy.

\paragraph{Definition:}
$x^* \in \mathcal{P}$, that is a basic feasible solution $\rightarrow \exists$
basis $B$ such that $x^* = (x^*_B, x^*_N) = (A^{-1}_B, 0)$.
The $j$-th non-basis direction is the vector $d^* \in \mathbb{R}^n$ (uniquely
determined), such that for $j \in N$, $d^*_N = \underbrace{e_j \in 
\mathbb{R}^{|N|}}_{\text{all zeros but the $j$-th position}}$ (non-basis
variables), $d^*_B = -A^{-1}_B A_{\cdot j}$.

\paragraph{Lemma: Given a bfs $x^*$ for the primal. Assume that $x^*$ is
non-degenerate ($\bar{b}$ is strictly positive in all components).
The $j$-th non-basis direction is feasible.}

\subparagraph{Proof:}
Let $d^*$ be the $j$-th non-basis direction. $\underbrace{Ad^* = 0}_{Ax^* = b
\text{, stays at the same kernel}} = A_B d^*_B + A_N d^*_N =
-A_B A^{-1}_B A_{\cdot j} + A_N e_j = -A_{\cdot j} + A_{\cdot j } = 0$.\\

$x^* + \lambda d^* \in \mathcal{P}$, $\forall \lambda \geq 0 \rightarrow
x^* + \lambda d^* \geq 0$ such that $\lambda \leq \frac{x^*_i}{-d^*_i}$,
$\forall i \in B$ such that $d^*_i < 0$.\\

In the non-basis,  $x^*_i + \lambda d^*$ is always positive because $d$ is the
unit vector. So, take $i \in B$:

\[
  x^*_i + \lambda d^*_i =
  \left\{
	\begin{array}{ll}
    \geq x^*_i \geq 0 \text{, }\forall i \in B \text{ such that }d_i \geq 0\\
	\geq x^*_i + \frac{x^*_i}{-d^*_i} d^*_i = 0 \text{, } \forall i \in B
	\text{ such that } d_i < 0.
    \end{array}
  \right.
\]

\paragraph{Observation: the $j$-th non-basis direction $d^*$ satisfies
$c^T d^* = \bar{c_j}$.}

\subparagraph{$c^T d^* = c^T_B d^*_B + c^T_N d^*_N = -c^T_B A^{-1}_B
A_{\cdot j} + c_j = \bar{c_j}$}

\paragraph{Theorem: Let $x^*$ be a bfs for the primal, with associated basis
$B$. Let $j \notin B$ st $\bar{c_j} < 0$. Assume $x^*$ is non-degenerate.
Let $d$ be the $j$-th non-basis direction:}
\begin{enumerate}
\item if $d_i \geq 0$ $\forall i \in B \rightarrow \displaystyle \min_{x
\in \mathcal{P}} c^{T} x = -\infty$.
\item if $d \ngeq 0$ (some components are smaller than zero),
then $\lambda^* = \min \{\frac{x^*_i}{-d_i} \mid i \in B \text{ st } d_i < 0 \}
= \max \{\lambda \geq 0 \mid x^* + \lambda d \in \mathcal{P} \}$.
\end{enumerate}

\subparagraph{Proof:}
\begin{enumerate}
\item $Ad =0 \rightarrow A(x^* + \lambda d) = b$, $\forall \lambda \geq 0$.
$c^T d = \bar{c_j} < 0$. $x^*$ is feasible so, $x^* \geq 0$. If $d \geq 0$
then $x^* + \lambda d \geq 0$, $\lambda \geq 0$. Then $x^* + \lambda d \in
\mathcal{P}$, $\forall \lambda > 0$, $c^T(x^* + \lambda d) = c^T x^* +
\lambda \bar{c_j} \xrightarrow[\lambda \to \infty]{} -\infty \rightarrow
\displaystyle \min_{x \in \mathcal{P}} c^{T} x = -\infty$.
\item $Ad = 0$, $x^*_i + \lambda d_i \geq 0$, $\lambda \geq 0 \rightarrow
\lambda \leq \frac{x^*_i}{-d_i}$, $\forall i \in B$, such that $d_i < 0$,
$\rightarrow \lambda^* = \max \{ \lambda > 0 \mid x^* + \lambda d
\in \mathcal{P} \}$.
\end{enumerate}

\paragraph{Theorem: Let $x^*$ be a bfs of $\mathcal{P}$ (which is not
necessarily non-degenerate). Suppose that there exists an index $j \in N$ st
$\bar{c_j} < 0$. Let $d$ be the $j$-th non basis direction. Assume that at
least one entry of $d$ is strictly negative ($d < 0$). Let $k \in B$, be an
index reaching the minimum in the definition of $\lambda^*$, i.e.,
$\lambda^* = \frac{x^*_k}{-d_k} = \displaystyle \min_{i \in \{1, \dots, n\}}
\frac{x_i}{-d_i}$. Let $\bar{B} = B\setminus \{ k\} \cup \{j\}$:}
\begin{enumerate}
\item $\bar{B}$ is a basis of $x^*$ w.r.t. to $\mathcal{P}$.
\item $x^* + \lambda^* d $ is the bfs associated with $\bar{B}$.
\end{enumerate}

\subparagraph{Proof:}

\begin{enumerate}
\item Proof of $\bar{B}$ be a basis:
\subitem $i \in \{1, \dots, n\} \rightarrow$ true
\subitem $|B| = n \rightarrow$ true.
\subitem Now, show $A^{-1}_B$ exists (and has full row rank).\\
$A_B$ is non singular $\iff A^{-1}_B A_{\bar{B}}$ is non singular.\\
Let $d$ be the $j$-th non-basis direction. $d_B = -A^{-1}_B A_{\cdot j}$,
$d_k \leq 0$, so, $A^{-1}_B A_{\bar{B}}$ is non singular, because
$(A^{-1}_B A_{\cdot j})_k \neq 0$.\\
\item From previous proof, $\bar{B}$ is a basis.
Define $y = x^* + \lambda^* d$. $\forall i \in N \setminus \{j\}$:
$y_i = x^*_i + \lambda^* \underbrace{d_i}_{d_i = 0 \text{; } d_j = 1} = 0
\rightarrow y_k = x^*_k + \lambda^* d_k = x^*_k + \frac{x^*_k}{-d_k} d_k = 0$.
\end{enumerate}

We have found a feasible solution $y$ to the system: $Az = b$, $z_i = 0
\underbrace{\forall i \notin \bar{B}}_{i \in N \setminus \{j\} \cup \{k\}}$.
This system has a unique solution: $\hat{z}$. $\hat{z}_B = A^{-1}_{\bar{B}} b$,
$\hat{z}_N = 0$. $\hat{z} = y$.

\subsection{Simplex Algorithm - primal setting}
\begin{enumerate}
\item Choose a bfs $x^*$ with basis $B$ and cost $c^* = c^T x^*$.
\item Compute the reduced cost: $\bar{c_j} = c_j - c^T_B A^{-1}_B A_{\cdot j}$,
$\forall j \in N$.
\item if $\bar{c} \geq 0$, stop $\rightarrow$ optimal!
\item Otherwise, determine $j \in N$, such that $\bar{c_j} < 0$.
\subitem if $d_B = - A^{-1}_B A_{\cdot j} \geq 0$, stop $\rightarrow$ unbounded!
\item Otherwise, compute $\lambda^* = \min_{i \in B} \{\frac{x^*_i}{-d_i}
\mid i \in B, d_i < 0 \}$ and index $k \in B$ such that $\lambda^* =
\frac{x^*_k}{-d_k}$.
\item Update the basis and the solutions:\\
$B = B\setminus \{k\} \cup \{j\}$ and set $x^*_k = 0$\\
$x^*_i = x^*_i + \lambda^* d_i$, $\forall i \in B\setminus \{k\}$\\
$x^*_j = \lambda^*$\\
$c^* = c^* + \lambda^* \bar{c_j}$
\end{enumerate}

\paragraph{Theorem:} For $\mathcal{P} \neq \emptyset$ and assuming that all
basic feasible solutions in $\mathcal{P}$ are non degenerate, then the simplex
algorithm terminates with a correct answer in finite time.
\subparagraph{Proof:}
Either the minimum value is $-\infty$ or the minimum value is finite.
Every pivot operation decreases the objective function value and leads us to a
new basis (finitely many), and no basis is visited twice.

\end{document}