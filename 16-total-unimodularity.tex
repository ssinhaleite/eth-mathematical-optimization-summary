\documentclass[main]{subfiles}
\begin{document}

%@@@@@@@@@@@@@@@@@@@@@@@@@@@@@@
% Main Topics: unimodular and totally unimodular matrices
% Total Unimodularity - 13.11.2017
% author: Vanessa Leite

\section{Total Unimodular Matrices}

\paragraph{Definition - Polyhedron integral} Let $P \subseteq \mathbb{R}^{n}$ be a polyhedron. P is integral if every "minimal wrt inclusion" face of $P$ contains integer points.

$P$ is a polyhedron and also a polytope. $P$ polytope $\rightarrow$ $P = conv(v^{1}, \dots, v^{t})$.
P is integral if the minimal faces (vertices) are integers, i.e, $v^{i} \in \mathbb{Z}^{n}$

\subparagraph{Fact from linear algebra - Cramer's rule}
$A \in \mathbb{Q}^{m \times n}$, regular, $b \in \mathbb{Q}^{m}$. $Ax = b \rightarrow \forall i = 1, dots, n$, $x_{i} = \frac{det[A^{i}}{det[A]}$, where $A^{i} = A_{\cdot l} \forall l \in \{1, \dots, n \} \setminus \{i\}$, $A^{i}_{\cdot i} = b$.

\paragraph{Definition - unimodular}
$A \in \mathbb{Z}^{m \times n}$ of full row rank is unimodular if the determinant of every basis of $A$ ($mxn$ regular submatrix) is equal to $\pm 1$.

\paragraph{Definition - totally unimodular}
$A \in \mathbb{Z}^{m \times n}$ is called totally unimodular (TU) if every square submatrix of $A$ has determinant $0$, $+1$, or $-1$.

\paragraph{Examples:}
\begin{itemize}
\item
$\begin{bmatrix}
3 & 5\\
1 & 2 \\
\end{bmatrix}$ is unimodular but not totally unimodular
\item $A$ is TU $\rightarrow A \in \{-1, 0, +1\}^{m \times n}$
\item $A \in \{0, 1\}^{2 \times w}$ is always TU: 
$\begin{bmatrix}
0 & 1 & 0 & 1 & \dots \\
1 & 1 & 0 & 0 & \dots \\
\end{bmatrix}$
\item $A \in \{-1, 0, 1\}^{2 \times n}$ is TU $\leftrightarrow$ it does not contain the submatrix 
$\begin{bmatrix}
\pm 1 & \pm 1 \\
\pm 1 & \pm -1  \\
\end{bmatrix}$
\item $A \in \{0,1\}^{3 \times n}$ is not always TU, for instance, determinant of 
$\begin{bmatrix}
1 & 1 & 0 \\
1 & 0 & 1 \\
0 & 1 & 1 \\
\end{bmatrix}$ is 2.
\end{itemize}

\paragraph{Exercises}
\begin{itemize}
\item $A$ is TU $\leftrightarrow [AI]$ is unimodular
\item $A$ is TU $\leftrightarrow$
$\begin{bmatrix}
A \\
-A \\
I \\
-I \\
\end{bmatrix}$ is TU
\item $A$ is TU $\leftrightarrow A^{T}$ is TU
\end{itemize}

\paragraph{Theorem - Unimodular}
Let $A \in \mathbb{Z}^{m \times n}$ of full row rank. $A$ is unimodular if and only if the family of polyhedra $P(b) = \{ x \in \mathbb{R}^{n} \mid Ax = b \}$ is integral $\forall b \in \mathbb{Z}^{m}$ such that $P(b) \neq \emptyset$.

\subparagraph{Proof}
\begin{itemize}
\item direction "$\rightarrow$", Assume $A$ is unimodular $\rightarrow$ let $b \in \mathbb{Z}^{m}$, $P(b) \neq \emptyset$. Take an extreme point $x^{*} \in P(b)$. $x^{*} = (x^{*}_{B}, x^{*}_{N})$ for basis B, $x^{*}_{B} = A^{-1}_{B}b$, $x^{*}_{N} = 0$.
$det[A_{B}] = \pm 1$ (because A is unimodular). From Cramer's rule $A_{B} \dot x^{*}_{B} = b$, $x^{*}_{B}$ is integral.

\item direction "$\leftarrow$", Assume $x^{*}$ is integral. Suppose $P(b)$ is integral, $\forall b \in \mathbb{Z}^{m}$ such that $P(b) \neq \emptyset$. Let $B \subseteq \{1, \dots, n\}$ be a basis. We want to show $det A_{B} \in \{-1, +1\}$.
Let $b = A_{B}z + \underbrace{e_{i}}_{ith unit vector}$, where $z$ is integer ($z \in \mathbb{Z}^{m}$ such that $z + A^{-1}_{B} e_{i} \geq 0$. Consider $P(b)$ and the extreme point $(x^{*}_{B}, x^{*}_{N})$, $x^{*}_{B} = A^{-1}_{B}b = z + A^{-1}_{B} e_{i}$ which by assumption is feasible for $P(b)$ by $z + A^{-1}_{B} e_{i} \geq 0$. $z + A^{-1}_{B} e_{i} \in \mathbb{Z}^{m} \rightarrow A^{-1}_{B} e_{i} \in \mathbb{Z}^{m} \rightarrow A^{-1}_{B}$ is integral $\rightarrow det(A_{B}) \in \{-1, +1\}$.
\end{itemize}

\paragraph{Theorem - Totally Unimodular}
Let $A \in \mathbb{Z}^{m \times n}$. $A$ is TU if and only if the family of polyhedra $P(b) = \{ x \in \mathbb{R}^{n} \mid Ax \leq b \}$ is integral $\forall b \in \mathbb{Z}^{m}$ such that $P(b) \neq \emptyset$.

\subparagraph{Proof}
$A$ is TU $\leftrightarrow \begin{bmatrix}
A & I\\
 & -I\\
\end{bmatrix}$ unimodular.

$P(b) = \{x \in \mathbb{R}^{n} \mid Ax \leq b \}$. Extreme points of $P(b)$ are in 1-1-correspondence with extreme points in $\{(x,y) \mid Ax + Iy = b, y \geq 0 \}$. From previous proof result, this result follows.

\paragraph{Theorem - combination property} $A \in \mathbb{Z}^{m \times n}$ is TU $\leftrightarrow$ for every subset $J$ of $\{1, \dots, n\{$ we can partition $J$ into $J_{1}$ and $J_{-1}$ such that $\mid \sum_{j \in J} A_{ij} - \sum_{j \in J_{-1}} A_{ij} \mid$ $\leq 1$ $\forall i \in \{1, \dots, m\}$.

\subparagraph{Proof}


\todo[inline]{supplementary: sections 19.1 and 19.2 from [sch86]}

\end{document}
