\documentclass[main]{subfiles}
\begin{document}

%@@@@@@@@@@@@@@@@@@@@@@@@@@@@@@
% Main Topics: extreme  rays, recession cones and representation of polyhedra 
% Representation of Polyhedra - 12.10.2017
% author: Vanessa Leite

\section{Representation of Polyhedra}
Definition: $Let \mathcal{P} = \{x \in \mathbb{R}^n \mid Ax \leq b\}$ be a 
polyhedron. $dim(\mathcal{P}) = n - dim(\{A_{i\cdot} \mid A_{i\cdot}x = b_i$, $
\forall x \in \mathcal{P} \})$ (implicitly linearly independent equations).

For $I \subseteq \{1, \dots, m\}$, $F = \{ x \in \mathcal{P} \mid A_{i\cdot} x = 
b$, $\forall i \in I \}$ is a face of $\mathcal{P}$.
Extreme case: $F = \emptyset$, $F = P$.\\

Vertices are faces of dimension $0$. Edges are faces of dimension $1$. Facets 
are faces of dimension $dim(\mathcal{P})-1$.\\

When $b = 0$, $\mathcal{P}$ is a cone. A cone is \textbf{pointed} if it does not 
contain a line.\\

For $\mathcal{P}$ and a point $y \in \mathcal{P}$, $\underbrace{rec}
_{\text{recession}}(y, \mathcal{P}) = \{ d \in \mathbb{R}^n \mid y + \lambda d 
\in \mathcal{P}$, $\forall \lambda \geq 0 \}$. If $\mathcal{P}$ is bounded then 
$rec(y, \mathcal{P}) = \{ 0 \}$, $\forall y \in \mathcal{P}$. \\

\paragraph{Observation:}
if $\mathcal{P} = \{x \in \mathbb{R}^n \mid Ax \leq b \}$, then the recession
cone $rec(y, \mathcal{P}) = \{d \in \mathbb{R}^n \mid Ad \leq 0 \}$, $\forall y
\in \mathcal{P}$. Therefore for every point $y \in \mathcal{P} \rightarrow
rec(y, \mathcal{P}) = rec(\mathcal{P}) = \{ d \in \mathbb{R}^n \}$.



\end{document}